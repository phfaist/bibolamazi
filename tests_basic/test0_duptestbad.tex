%
% For testing purposes, I have edited this file manually and modified the header. Now,
% the duplicates filter should see that the header has been altered or removed (see
% below), and should not overwrite the file.
%
%



%
% --- altered header ---
%
% ----- NOTE: THIS FILE WAS AUTOMATICALLY GENERATED BY bibolamazi SCRIPT!
% ---------      ANY CHANGES WILL BE LOST!
%
% ----File automatically generated by bibolamazi's `duplicates` filter.
%
% You should include this file in your main LaTeX file with the command
%
%   
%
% NOTE: THIS FILE WAS AUTOMATICALLY GENERATED BY bibolamazi SCRIPT!
%       ANY CHANGES WILL BE LOST!
%
% You should include this file in your main LaTeX file with the command
%
%   
%
% NOTE: THIS FILE WAS AUTOMATICALLY GENERATED BY bibolamazi SCRIPT!
%       ANY CHANGES WILL BE LOST!
%
% You should include this file in your main LaTeX file with the command
%
%   
%
% NOTE: THIS FILE WAS AUTOMATICALLY GENERATED BY bibolamazi SCRIPT!
%       ANY CHANGES WILL BE LOST!
%
% You should include this file in your main LaTeX file with the command
%
%   \input{dupfile.tex}
%
% in your document preamble.
%



%
% The following will define the command \bibalias{<alias>}{<source>}, which will make
% the command \cite[..]{<alias>} the same as doing \cite[..]{<source>}.
%
% This code has been copied and adapted from
%    http://tex.stackexchange.com/questions/37233/
%

\makeatletter
% \bibalias{<alias>}{<source>} makes \cite{<alias>} equivalent to \cite{<source>}
\newcommand\bibalias[2]{%
  \@namedef{bibali@#1}{#2}%
}

\newtoks\biba@toks
\let\bibalias@oldcite\cite
\renewcommand\cite[2][]{%
  \biba@toks{\bibalias@oldcite#1}%
  \def\biba@comma{}%
  \def\biba@all{}%
  \@for\biba@one:=#2\do{%
    \@ifundefined{bibali@\biba@one}{%
      \edef\biba@all{\biba@all\biba@comma\biba@one}%
    }{%
      \PackageInfo{bibalias}{%
        Replacing citation `\biba@one' with `\@nameuse{bibali@\biba@one}'
      }%
      \edef\biba@all{\biba@all\biba@comma\@nameuse{bibali@\biba@one}}%
    }%
    \def\biba@comma{,}%
  }%
  \edef\biba@tmp{\the\biba@toks{\biba@all}}%
  \biba@tmp
}
\makeatother


%
% Now, declare all the alias keys.
%

\bibalias{del2011thermodynamic}{delRio2011Nature}
\bibalias{1211.3141}{1211.3141v2}
\bibalias{1306.0516}{1306.0516v2}
\bibalias{quant-ph/0512258}{quant-ph/0512258v2}



%
% in your document preamble.
%



%
% The following will define the command \bibalias{<alias>}{<source>}, which will make
% the command \cite[..]{<alias>} the same as doing \cite[..]{<source>}.
%
% This code has been copied and adapted from
%    http://tex.stackexchange.com/questions/37233/
%

\makeatletter
% \bibalias{<alias>}{<source>} makes \cite{<alias>} equivalent to \cite{<source>}
\newcommand\bibalias[2]{%
  \@namedef{bibali@#1}{#2}%
}

\newtoks\biba@toks
\let\bibalias@oldcite\cite
\renewcommand\cite[2][]{%
  \biba@toks{\bibalias@oldcite#1}%
  \def\biba@comma{}%
  \def\biba@all{}%
  \@for\biba@one:=#2\do{%
    \@ifundefined{bibali@\biba@one}{%
      \edef\biba@all{\biba@all\biba@comma\biba@one}%
    }{%
      \PackageInfo{bibalias}{%
        Replacing citation `\biba@one' with `\@nameuse{bibali@\biba@one}'
      }%
      \edef\biba@all{\biba@all\biba@comma\@nameuse{bibali@\biba@one}}%
    }%
    \def\biba@comma{,}%
  }%
  \edef\biba@tmp{\the\biba@toks{\biba@all}}%
  \biba@tmp
}
\makeatother


%
% Now, declare all the alias keys.
%

\bibalias{del2011thermodynamic}{delRio2011Nature}
\bibalias{1211.3141}{1211.3141v2}
\bibalias{1306.0516}{1306.0516v2}
\bibalias{quant-ph/0512258}{quant-ph/0512258v2}



%
% in your document preamble.
%



%
% The following will define the command \bibalias{<alias>}{<source>}, which will make
% the command \cite[..]{<alias>} the same as doing \cite[..]{<source>}.
%
% This code has been copied and adapted from
%    http://tex.stackexchange.com/questions/37233/
%

\makeatletter
% \bibalias{<alias>}{<source>} makes \cite{<alias>} equivalent to \cite{<source>}
\newcommand\bibalias[2]{%
  \@namedef{bibali@#1}{#2}%
}

\newtoks\biba@toks
\let\bibalias@oldcite\cite
\renewcommand\cite[2][]{%
  \biba@toks{\bibalias@oldcite#1}%
  \def\biba@comma{}%
  \def\biba@all{}%
  \@for\biba@one:=#2\do{%
    \@ifundefined{bibali@\biba@one}{%
      \edef\biba@all{\biba@all\biba@comma\biba@one}%
    }{%
      \PackageInfo{bibalias}{%
        Replacing citation `\biba@one' with `\@nameuse{bibali@\biba@one}'
      }%
      \edef\biba@all{\biba@all\biba@comma\@nameuse{bibali@\biba@one}}%
    }%
    \def\biba@comma{,}%
  }%
  \edef\biba@tmp{\the\biba@toks{\biba@all}}%
  \biba@tmp
}
\makeatother


%
% Now, declare all the alias keys.
%

\bibalias{del2011thermodynamic}{delRio2011Nature}
\bibalias{1211.3141}{1211.3141v2}
\bibalias{1306.0516}{1306.0516v2}
\bibalias{quant-ph/0512258}{quant-ph/0512258v2}



%
% in your document preamble.
%



%
% The following will define the command \bibalias{<alias>}{<source>}, which will make
% the command \cite[..]{<alias>} the same as doing \cite[..]{<source>}.
%
% This code has been copied and adapted from
%    http://tex.stackexchange.com/questions/37233/
%

\makeatletter
% \bibalias{<alias>}{<source>} makes \cite{<alias>} equivalent to \cite{<source>}
\newcommand\bibalias[2]{%
  \@namedef{bibali@#1}{#2}%
}


%
% Note: The `\cite` command provided here does not accept spaces in/between its
% arguments. This might be tricky, since revTeX does accept those spaces. You
% can work around by using LaTeX comments which automatically remove the
% following space after newline, in the following way:
%
%    \cite{key1,%
%          key2,%
%          key3%
%    }
%
% Make sure you don't add space between the comma and the percent sign.
%

\newtoks\biba@toks
\let\bibalias@oldcite\cite
\def\cite{%
  \@ifnextchar[{%
    \biba@cite@optarg%
  }{%
    \biba@cite{}%
  }%
}
\newcommand\biba@cite@optarg[2][]{%
  \biba@cite{[#1]}{#2}%
}
\newcommand\biba@cite[2]{%
  \biba@toks{\bibalias@oldcite#1}%
  \def\biba@comma{}%
  \def\biba@all{}%
  \@for\biba@one:=#2\do{%
    \@ifundefined{bibali@\biba@one}{%
      \edef\biba@all{\biba@all\biba@comma\biba@one}%
    }{%
      \PackageInfo{bibalias}{%
        Replacing citation `\biba@one' with `\@nameuse{bibali@\biba@one}'
      }%
      \edef\biba@all{\biba@all\biba@comma\@nameuse{bibali@\biba@one}}%
    }%
    \def\biba@comma{,}%
  }%
  \edef\biba@tmp{\the\biba@toks{\biba@all}}%
  \biba@tmp
}
\makeatother


%
% Now, declare all the alias keys.
%

\bibalias{del2011thermodynamic}{delRio2011Nature}
\bibalias{gour_measuring_2009}{gour_measuring_2009_dupl}


